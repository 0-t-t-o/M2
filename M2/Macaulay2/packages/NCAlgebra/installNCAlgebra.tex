\documentclass[10pt]{article}
\usepackage{amsmath}
\usepackage{amsthm}
\usepackage{amssymb}
\usepackage{latexsym}
\usepackage{enumerate}
\usepackage{verbatim}
\usepackage[colorlinks=true,linkcolor=blue,urlcolor=blue]{hyperref}

% put in BEFORE fancyhdr
\usepackage[margin=1in]{geometry}

\parindent=0pt
\parskip=4pt

\begin{document}

\begin{center}\textsc{Installation instructions for NCAlgebras}\end{center}

These instructions have been checked for Linux (specifically Ubuntu) as well as Mac OS X up
through Sierra.  We have yet to install it natively on a Windows machine, though with the new Linux Subsystem
on Windows it may be possible.

\begin{enumerate}

\item Regardless of your platform, the first thing to do is install Macaulay2.  This can be done by following
the instructions located at \href{http://www.macaulay2.com}{\texttt{www.macaulay2.com}}.  The \texttt{NCAlgebra}
package is also distributed with the most recent version of Macaulay2 (v1.11), but not earlier versions.
Instructions for installing \texttt{NCAlgebra} on earlier versions are also included below.

\item Install Common Lisp (also known as CLisp).  On a Linux machine this is usually accomplished via the package manager
built into your distribution.  For example, on Ubuntu the command \texttt{sudo apt-get install clisp} should
do the trick.

On a Mac, one must first install homebrew.  Instructions for installing homebrew are on their webpage \href{https://brew.sh}{\texttt{brew.sh}}.
This installation may also require you to install the XCode Command Line Developer Tools.  Instructions how to accomplish
this are located \href{http://osxdaily.com/2014/02/12/install-command-line-tools-mac-os-x/}{here}.  Once homebrew is installed,
execute \texttt{brew install clisp} at a command prompt.

\item Download the Bergman system \href{http://servus.math.su.se/bergman/}{here}.  Extract the \texttt{tar.gz} file to a directory accessible
by all the users that wish to use the system.  In what follows, I will call the location of this directory \texttt{<bergmanroot>}.

\item Open a terminal and navigate to the \texttt{<bergmanroot>} directory.  Here the instructions for Linux and Mac diverge a bit:
\begin{itemize}
\item Linux: Change to the directory \texttt{<bergmanroot>/scripts/clisp/unix}.  Execute the command:

\begin{center}\texttt{./mkbergman -auto}
\end{center}
This command will build the bergman executable.  Move to step 5.
\item Mac: Things are a bit more complicated.  The issue is that CLisp seems to have broken how CLisp generates native executables on a Mac in CLisp 2.49,
or at the very least the more recent versions of the Mac OS do not understand how to run the executables that are generated by CLisp.  To get around
this, follow these steps:
\begin{enumerate}
\item Change to the directory \texttt{<bergmanroot>/auxil/clisp}.  Edit the file \texttt{bm-tail-cl.lsp} in a text editor.  You will see the lines

\begin{verbatim}
;;(SAVEINITMEM "lispinit.mem" :INIT-FUNCTION...
(SAVEINITMEM "bergman.exe" :INIT-FUNCTION...
\end{verbatim}

in the file.  In CLisp, \texttt{;;} indicates a comment.  Switch the lines that are commented; that is, place \texttt{;;} on the front of the
second line and take the \texttt{;;} off of the first line.  Save your changes.

\item Change to the \texttt{<bergmanroot>/scripts/clisp2.29/unix/} directory and execute the command:
\begin{center}
\texttt{./mkbergman -auto}
\end{center}
This command will build the bergman executable.

\item Finally change the directory to \texttt{<bergmanroot>/bin/clisp/unix}.  In a text editor edit
the \texttt{bergman} file there.  This is a shell script which loads the necessary files to start the bergman
executable.  However there is a change that must be made to this file as well.  Here, \texttt{\#} denotes a comment.
By default the third line is active and the second one is commented.  Switch these around as well.  Save your changes.
\end{enumerate}
\end{itemize}

\newpage 

\item Make sure the \texttt{bergman} executable can be run from any command prompt.  One way to do this is to add a symbolic link 
of the bergman executable script to \texttt{/usr/local/bin} (or any other directory already on your path).  This command
may look something like:
\begin{center}
\texttt{ls -s <bergmanroot>/bin/clisp/unix/bergman /usr/local/bin/bergman}
\end{center}
provided that you followed the steps above to generate the \texttt{bergman} executable.  Note that in the above
command, the \emph{full} path (from the root directory) to \texttt{<bergmanroot>} must be given.

\item The most recent release of Macaulay2 (v1.11) contains the \texttt{NCAlgebra} package as part of the distribution.
If you are running an older version of Macaulay2, it is a good idea to upgrade in case some bugs have
been fixed and improvements made.  If you must use an older version, \texttt{NCAlgebra} is available on Frank Moore's webpage
\href{http://users.wfu.edu/moorewf}{here}.

\item Start Macaulay2.  Run the command \texttt{installPackage "NCAlgebra"} at the Macaulay2 prompt.

\item Test your installation by running the following commands in Macaulay2:

\begin{verbatim}
needsPackage "NCAlgebra"
R = fourDimSklyanin(QQ,{a,b,c,d})
hilbertBergman(R, DegreeLimit => 6)
\end{verbatim}

If no errors are generated, then you have successfully installed the system.  Enjoy!

\end{enumerate}

\end{document}

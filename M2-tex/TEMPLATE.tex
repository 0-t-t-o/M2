\documentclass{article}
%\usepackage{amstex}
\usepackage{amsmath}
\usepackage{amsthm}
\usepackage{amssymb}
% other options to documentstyle: twocolumn
\input xy \xyoption{all} \CompileMatrices
\newtheorem{theorem}	{Theorem}	[section]
\newtheorem{lemma}	[theorem]	{Lemma}
\newtheorem{cor}	[theorem]	{Corollary}
\newtheorem{prop}	[theorem]	{Proposition}
\newtheorem{example}	[theorem]	{Example}
\newtheorem{definition}	[theorem]	{Definition}
\newtheorem{convention}	[theorem]	{Convention}
\newtheorem{note}	[theorem]	{Remark}
\begin{document}
\author{John Doe \\ University of Illinois at Chicago
	\and
	Daniel R. Grayson \\ University of Illinois at Urbana-Champaign
	\thanks{Supported by the NSF.}
	}
\date{July 10, 1998}
\title{Martingales and the Common Flicker}
\maketitle

\section{Introduction}
Our main goal is to prove Theorem \ref{thm1}.

\begin{theorem} \label{thm1}
The following diagram is interesting.
$$
\xymatrix{
M_J \ar[r]^{\theta_J}  \ar[dr]^{\theta_J} & 
	N(\tilde{J}) \ar[d]^{\phi_{\tilde{J}}} \\
& N(\tilde{J}) 
}
$$
\end{theorem}

\begin{proof}
The proof follows from the usual induction, as in \cite{MR94g:19005}.
\end{proof}

Here is some {\sl Macaulay 2} stuff.
<<<R = ZZ[a..f]>>>
and some more, too:
<<<100!>>>

\bibliographystyle{plain}
\bibliography{papers}
\end{document}
